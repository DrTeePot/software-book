\section{Design Patterns}

\subsection{Types of advantages}
\begin{description}
    \item[Ease of use]: Creating a subclass based on the base class is easy
        to manage, it only requires the subclass to provide either a
        concrete implementation of an abstract method, or a more specific
        implementation of a method that is allowed by the base class.
    \item[open-closed principle]: the design philosophy is to be ``Open to
        Extension, Closed to Modification''.  Here the base class provides a
        framework code that can be extended through inheritance to allow
        instances of new-but-related classes with new fields and methods.
        This allows designers to be achieve new design requirements without
        having to change the code of the base class, or the algorithms and
        data structures which operate on instances of the base class.
    \item[reusability of the code]: Duplication of code is minimized through
        the use of base classes to establish a common pattern among
        subclasses. ``Inheritance'' means subclasses only need to write down
        overwritten methods for the base class to use.
    \item[flexibility]: Due to its reliance on
        inheritance to establish the behavior of subclasses, usually
        subclasses are restrained to inheriting from one base class at a
        time. As a result, any subclass is restrained to follow the general
        behavior established by its parent base class.
    \item[maintainability]: Reading the flow of code is
        difficult due to the disjoint nature of the code. Code within the
        base class only provides the steps that its subclasses will take,
        whereas subclasses only contain details for a step. In addition,
        additional functionality is difficult to implement due to the tight
        coupling between classes. Adding features to the base class
        requires the change to be applied to all related subclasses.
\end{description}

